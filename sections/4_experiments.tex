\section{Results and Philosophical Analysis}
\label{sec:analysis}

Our evaluation yields a significant discovery: \textbf{Consistency is being used as a shield against complexity.}

\subsection{The Prevalence of the Static Video Trap}
As shown in Table~\ref{tab:main_results}, VideoCrafter2 exhibits the most severe case of the Static Video Trap. It achieves a Subject Consistency of 0.96, yet its Background Diversity remains at a mere 0.15. This numerical disparity confirms that the model avoids scene transitions entirely to maintain a high identity score. We argue that this represents a local minimum in the optimization landscape of diffusion models.

\subsection{The Failure of Temporal Smoothness}
AnimateDiff demonstrates a different failure mode, characterized by low Cut-Transition Sharpness (0.25). In multi-shot prompts, rather than producing distinct cinematic cuts, this model "morphs" the geometry of Shot 1 into Shot 2. While this results in high temporal smoothness scores in traditional metrics like FVD, it is semantically incorrect and visually jarring for narrative storytelling.

\subsection{Implications for Future Research}
The current benchmarking results suggest that "more data" or "larger models" may not solve the Static Video Trap. Instead, there is a need for architectural innovations that explicitly decouple the \textit{identity prior} from the \textit{dynamic prior}. Our Dynamic-MSV-Bench provides the necessary granularity to track progress in this decoupled generation space.
