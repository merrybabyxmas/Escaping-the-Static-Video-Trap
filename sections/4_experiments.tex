\section{Experiments and Unveiling the Blind Spots}
\label{sec:experiments}

Our evaluation on Dynamic-MSV-Bench reveals a systematic failure in current video generation architectures. To properly diagnose these failures, we categorized the evaluated baselines into two distinct groups: \textbf{Group 1: Foundation T2V Models} and \textbf{Group 2: Specialized T2MSV Frameworks}.

\begin{figure*}[ht]
    \centering
    \begin{subfigure}{0.48\linewidth}
        \includegraphics[width=\linewidth]{figures/radar_chart_set_a.pdf}
        \caption{Track S: Semantic Leap}
    \end{subfigure}\hfill
    \begin{subfigure}{0.48\linewidth}
        \includegraphics[width=\linewidth]{figures/radar_chart_set_b.pdf}
        \caption{Track M: Motion Continuity}
    \end{subfigure}
    \caption{Radar Charts of Model Performance. In \textbf{Track S}, an ideal model would form a large diamond. In \textbf{Track M}, where diversity and sharpness should be minimized, an ideal model would form a wide horizontal "bowtie" (high consistency and high alignment). Current models fail to achieve these target geometries.}
    \label{fig:radar}
\end{figure*}

\subsection{Group 1: Foundation T2V Models and the Static Trap}
Foundation models (e.g., CogVideoX, LTX-Video, SVD) exhibit the raw limits of spatiotemporal priors. They achieve high Subject Consistency (>0.80) but catastrophically low Background Diversity (<0.10) in Track S. Their upgraded \textbf{DSA} scores remain near $0.02$, proving they generate a single static scene regardless of narrative instructions. This results in the uniform probability distribution seen in Fig.~\ref{fig:dsa_heatmaps}(A).

\subsection{Group 2: T2MSV Frameworks and the Trade-off Dilemma}
Specialized frameworks like StoryDiffusion successfully break out of the Static Trap, achieving higher Background Diversity in Track S. However, this dynamism comes at the severe cost of identity preservation. Their Subject Consistency plummets, leading to \textit{Identity Amnesia}, visualized by the Pareto frontier in Figure~\ref{fig:scatter}. 

\begin{figure}[ht]
    \centering
    \includegraphics[width=\linewidth]{figures/tradeoff_scatter.pdf}
    \caption{The trade-off between consistency and dynamics on Track S. Current SOTA models cluster in the top-left (Static Trap) or bottom-right (Amnesia), leaving the top-right goal vacant. Dot size represents Diagonal Semantic Alignment (DSA).}
    \label{fig:scatter}
\end{figure}

\subsection{The "Double-Kill" and the Context Paradox}
Our evaluation exposes a fascinating paradox: a metric's value is deeply context-dependent. For instance, CogVideoX exhibits exceptionally low Background Diversity. In Track M, traditional evaluators might misinterpret this as excellent spatial continuity. However, our cross-scenario analysis reveals the truth: the same model exhibits the same low diversity in Track S, proving it is merely trapped in static generation. Without the dual-track perspective of Track S and Track M, this critical failure would remain masked.

\begin{figure}[ht]
    \centering
    \includegraphics[width=\linewidth]{figures/fig4_filmstrip.pdf}
    \caption{Qualitative comparison on Track M. The "Static Trap" (Top Row) mimics continuity through total lack of motion (DSA $\approx 0$). Specialized frameworks (Bottom Row) attempt motion but ignore the specific "Panning" vs. "Zooming" instructions (Low DSA). Both current paradigms fail to execute controlled multi-shot motion.}
    \label{fig:qualitative}
\end{figure}

\begin{table*}[t]
\centering
\caption{Main Results on Track S (Semantic Leap). (Golden Rule: All metrics should be $\uparrow$)}
\label{tab:main_results_track_s}
\resizebox{\textwidth}{!}{
\begin{tabular}{ll|cccc}
\toprule
\textbf{Category} & \textbf{Method} & \textbf{Subj. Cons.} $\uparrow$ & \textbf{BG Div.} $\uparrow$ & \textbf{DSA} $\uparrow$ & \textbf{Sharpness} $\uparrow$ \\
\midrule
Foundation & CogVideoX & 0.90 & 0.07 & 0.04 & 0.26 \\
Foundation & SVD & 0.82 & 0.14 & 0.02 & 0.10 \\
Framework & StoryDiffusion & 0.44 & 0.42 & 0.22 & 0.70 \\
Framework & FreeNoise & 0.53 & 0.35 & 0.03 & 0.89 \\
\bottomrule
\end{tabular}}
\end{table*}

\begin{table*}[t]
\centering
\caption{Main Results on Track M (Motion Continuity). (Golden Rule: Cons. $\uparrow$, Div. $\downarrow$, DSA $\uparrow$, Sharpness $\downarrow$)}
\label{tab:main_results_track_m}
\resizebox{\textwidth}{!}{
\begin{tabular}{ll|cccc}
\toprule
\textbf{Category} & \textbf{Method} & \textbf{Subj. Cons.} $\uparrow$ & \textbf{BG Div.} $\downarrow$ & \textbf{DSA} $\uparrow$ & \textbf{Sharpness} $\downarrow$ \\
\midrule
Foundation & CogVideoX & 0.94 & 0.05 & 0.00 & 0.43 \\
Foundation & SVD & 0.86 & 0.11 & 0.01 & 0.05 \\
Framework & StoryDiffusion & 0.31 & 0.52 & 0.02 & 0.67 \\
Framework & FreeNoise & 0.73 & 0.20 & 0.05 & 1.00 \\
\bottomrule
\end{tabular}}
\end{table*}
