\section{Related Work}
\label{sec:related}

\subsection{Foundation Models vs. Specialized Frameworks}
The field has bifurcated into \textbf{Foundation T2V Models} (e.g., CogVideoX~\cite{yang2024cogvideox}, LTX-Video~\cite{HaCohen2024LTXVideo}, SVD~\cite{blattmann2023stable}) and \textbf{Specialized T2MSV Frameworks} (e.g., StoryDiffusion~\cite{zhou2024storydiffusion}, DirecT2V~\cite{hong2023large}, Mora~\cite{yuan2024mora}). While foundation models learn raw spatiotemporal priors, they often lack the explicit control needed for multi-shot narrative. Specialized frameworks attempt to enforce consistency using reference attention or agentic pipelines, but our evaluation reveals they frequently suffer from \textit{prompt bleeding} and \textit{identity amnesia}.

\subsection{The Flaws of Legacy Evaluation in T2MSV}
Current benchmarks rely on three types of metrics that we argue are ill-suited for T2MSV:

\textbf{1. The Global Similarity Paradox:} Relying on holistic CLIP similarity penalizes correct behavior. If a model accurately executes a semantic shift (Track S), the holistic similarity drops, causing the model to be penalized for following instructions. We resolve this by decoupling the subject via DINOv2~\cite{oquab2023dinov2} masking.

\textbf{2. The Uniform Continuity Bias:} Standard temporal metrics treat intentional cinematic "cuts" as visual artifacts. We introduce Track-aware Sharpness, rewarding high sharpness only when a shot transition is explicitly prompted.

\textbf{3. VLM-as-a-Judge Limits:} Language models are bad at catching pixel-level "morphing"—where objects melt into one another instead of cutting cleanly. Our mathematical \textbf{DSA} metric provides a more rigorous, objective measure of independent adherence.
