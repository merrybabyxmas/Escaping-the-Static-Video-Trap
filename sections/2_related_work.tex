\section{Related Work}
\label{sec:related}

\subsection{Specialized T2MSV Frameworks}
Recent research has begun to address the complexities of narrative video. StoryDiffusion~\cite{zhou2024storydiffusion} and DirecT2V~\cite{hong2023large} attempt to maintain character consistency across multiple generations using reference attention or LLM-driven directors. However, as our experiments show, these models often suffer from \textbf{Prompt Bleeding}, where semantic information from Shot 1 contaminated Shot 2, preventing sharp narrative transitions.

\subsection{The Flaws of Legacy Evaluation}
Current benchmarks like VBench~\cite{huang2023vbench} rely on three types of metrics that we argue are ill-suited for T2MSV:

\textbf{1. Holistic Image Similarity:} StoryDiffusion~\cite{zhou2024storydiffusion} uses average CLIP similarity between all frames to prove character consistency. This creates the \textit{Global Similarity Paradox}: if a model correctly changes the background as prompted, the holistic similarity \textit{decreases}. Thus, models are penalized for correct behavior. Our framework resolves this by using DINOv2~\cite{oquab2023dinov2} based subject masking to isolate identity from dynamics.

\textbf{2. Uniform Continuity Bias:} Standard temporal metrics reward sub-pixel smoothness across all frames. In T2MSV, cinematic grammar requires intentional "cuts" or discontinuities. Legacy metrics treat these sharp transitions as visual artifacts. We introduce Track-aware Sharpness, which rewards high sharpness only when a shot transition is explicitly prompted.

\textbf{3. VLM-as-a-Judge Limits:} While DirecT2V~\cite{hong2023large} employs GPT-4V for narrative QA, language models are notoriously bad at catching pixel-level "morphing" artifacts—where one object slowly dissolves into another instead of cutting. Our mathematical \textbf{DSA} metric, utilizing CLIP~\cite{radford2021learning}, provides a more rigorous, objective measure of shot-level independent adherence.
