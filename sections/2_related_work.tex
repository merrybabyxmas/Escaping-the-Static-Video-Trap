\section{Related Work}
\label{sec:related}

\subsection{Foundation Models vs. Specialized Frameworks}
The field has bifurcated into \textbf{Foundation T2V Models} (e.g., CogVideoX~\cite{yang2024cogvideox}, LTX-Video~\cite{HaCohen2024LTXVideo}, SVD~\cite{blattmann2023stable}) and \textbf{Specialized T2MSV Frameworks} (e.g., StoryDiffusion~\cite{zhou2024storydiffusion}, DirecT2V~\cite{hong2023large}, Mora~\cite{yuan2024mora}). While foundation models learn powerful raw spatiotemporal priors, they critically lack the explicit control needed for multi-shot narrative. Specialized frameworks attempt to enforce consistency using reference attention or agentic pipelines; however, our evaluation reveals they frequently suffer from \textit{prompt bleeding} and \textit{identity amnesia}.

\subsection{The Flaws of Legacy Evaluation in T2MSV}
Current benchmark suites rely on three paradigms that we argue are fundamentally ill-suited for the cinematic grammar of T2MSV:

\paragraph{The Global Similarity Paradox.} Relying on holistic CLIP similarity penalizes correct generative behavior. If a model accurately executes a semantic shift (Track S), the holistic structural similarity artificially drops, causing the model to be penalized for strictly following instructions. We resolve this by decoupling the subject via DINOv2~\cite{oquab2023dinov2} masking.

\paragraph{The Uniform Continuity Bias.} Standard temporal coherence metrics treat all inter-frame differences as errors. In T2MSV, intentional cinematic "cuts" are mandatory. We introduce Track-aware Sharpness, rewarding high perceptual divergence exclusively when a shot transition is prompted.

\paragraph{VLM-as-a-Judge Limits.} Vision-Language Models (VLMs) often lack the granular spatial sensitivity required to detect pixel-level artifacts, such as "morphing"—where objects unnaturally fuse into one another instead of executing clean cinematic cuts. To overcome this, our mathematical \textbf{DSA} metric provides a rigorous, objective measure of independent prompt adherence.