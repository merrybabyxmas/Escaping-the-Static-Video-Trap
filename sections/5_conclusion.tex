\section{Conclusion}
\label{sec:conclusion}

In this paper, we introduced \textbf{Dynamic-MSV-Bench} and a novel 4D decoupled evaluation framework to rigorously address the systemic blind spots in Text-to-Multi-Shot Video (T2MSV) generation. By distinguishing between \textbf{Track S (Semantic Leap)} and \textbf{Track M (Motion Continuity)}, and by formally introducing \textbf{Diagonal Semantic Alignment (DSA)}, we have mathematically and qualitatively exposed the pervasive \textbf{Static Video Trap} and the \textbf{Identity Amnesia} phenomenon. Our comprehensive empirical evaluation reveals a fundamental "Double-Kill" dilemma: current monolithic paradigms are structurally constrained from achieving the simultaneous high-consistency, high-dynamics regime required for professional cinematic narratives. To overcome this, the field must pivot toward \textbf{Decoupled Dynamics}—generative architectures that strictly isolate identity preservation priors from narrative-driven motion priors. We believe our benchmark provides the essential diagnostic tool to track and catalyze this architectural paradigm shift toward truly dynamic, multi-shot AI video generation.