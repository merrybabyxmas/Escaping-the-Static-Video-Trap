\section{Conclusion}
\label{sec:conclusion}

In this paper, we introduced the \textbf{Dynamic-MSV-Bench} and a novel 4-dimensional evaluation framework to expose the critical limitations of current video generation models. Our comprehensive evaluation of both General T2V models (e.g., VideoCrafter2, Open-Sora) and Dedicated T2MSV models (e.g., StoryDiffusion, FreeNoise, DirecT2V) strictly followed their official implementations. 

The results unequivocally validate our core hypothesis: existing architectures are caught in a "Double-Kill" dilemma. Models either overfit to the \textit{identity prior}, resulting in the \textbf{Static Video Trap} where characters remain consistent but scenes refuse to change; or they attempt temporal extension through noise manipulation, falling into the \textbf{Morphing Trap} where shots melt into one another rather than cutting sharply.

These findings highlight that incremental improvements or prompt engineering cannot solve the fundamental entanglement of motion and identity within current diffusion mechanisms. To achieve true multi-shot narrative video generation, the field must move toward \textbf{Decoupled Dynamics}—where subject identity is strictly preserved without suppressing dynamic scene transitions or cinematic cuts. We believe our benchmark provides the essential compass for developing such decoupled architectures in future research.